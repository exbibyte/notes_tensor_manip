\documentclass[8pt,letter]{article}
 
%% \usepackage[fleqn]{amsmath}
\usepackage[margin=1in]{geometry}
\usepackage{amsmath,amsfonts,amsthm,bm}
\usepackage{breqn}
\usepackage{amsmath}
\usepackage{mathtools}
\usepackage{amssymb}
\usepackage{tikz}
\usepackage[ruled,vlined,linesnumbered,lined,boxed,commentsnumbered]{algorithm2e}
\usepackage{siunitx}
\usepackage{graphicx}
\usepackage{subcaption}
%% \usepackage{datetime}
\usepackage{multirow}
\usepackage{multicol}
\usepackage{mathrsfs}
\usepackage{fancyhdr}
\usepackage{fancyvrb}
\usepackage{parskip} %turns off paragraph indent
\pagestyle{fancy}

\usepackage{xcolor}
\usepackage{mdframed}

\usepackage[small]{titlesec}

\usetikzlibrary{arrows}

\DeclareMathOperator*{\argmin}{argmin}
\newcommand*{\argminl}{\argmin\limits}

\newcommand{\mathleft}{\@fleqntrue\@mathmargin0pt}
\newcommand{\R}{\mathbb{R}}
\newcommand{\Z}{\mathbb{Z}} 
\newcommand{\N}{\mathbb{N}}
\newcommand{\ppartial}[2]{\frac{\partial #1}{\partial #2}}
\newcommand{\p}{\partial}
\newcommand{\te}[1]{\text{#1 }}
\newcommand{\norm}[1]{\|#1\|}

\setcounter{MaxMatrixCols}{20}

% remove excess vertical space for align equations
\setlength{\abovedisplayskip}{0pt}
\setlength{\belowdisplayskip}{0pt}
\setlength{\abovedisplayshortskip}{0pt}
\setlength{\belowdisplayshortskip}{0pt}

\newtheorem{theorem}{Theorem}[section]
\newtheorem{corollary}{Corollary}[theorem]
\newtheorem{lemma}[theorem]{Lemma}

% \newtheorem{mdtheorem}{Theorem}
% \newenvironment{theorem}
% {\begin{mdframed}[
%   backgroundcolor=green!10,
%   topline=false,
%   rightline=false,
%   bottomline=false,
%   leftline=false
%   ]\begin{mdtheorem}}
%     {\end{mdtheorem}\end{mdframed}}

\begin {document}

\lhead{Notes - Tensor/Array Manipulations, Bill (Yuan) Liu}

% \begin{align}
%   \nabla f_{k+1}^T p_k & \geq c_2 \nabla f_k^T p_k\\
%   \frac{\partial \phi(\alpha_k)}{\partial \alpha_k} & \geq \frac{\partial \phi(0)}{\partial \alpha_k}\\
%   &c_1,\alpha_k \in (0,1)\\
%   &0<c_1<c_2<1
% \end{align}

\begin{multicols*}{2}

  \section{Einops}

  Reference: https://github.com/arogozhnikov/einops
  
  \subsection{Features}
  
  \begin{itemize}
  \item self-documenting notation for layouts of input and output arrays
  \item low number of backend functions to implement
  \item focus on data rearrangements and simple transformations (axes reordering, decomposition, composition, reduction, repeats)
  \item focus on 1 tensor/array transformations
  \item notation uses strings
  \item supported notations: named axis, anonymous axis, unitary axis, ellipsis, (de)compose parenthesis
  \item supports a list of arrays as input with implied additional outer dimension corresponding to the list
  \item inferrable dimension sizes, given partial info as parameters
  \item hide backend framework inconsistency of notations for common array rearrangement operations
  \item use of proxy classes for specific backends
  \item caching of tensor type map to backend type for performance
  \item caching of patterns and axes
  \item caching of patterns, axes, and input shape: compute unknown axis sizes and shape verification on first time, otherwise reuse sequence of commands previously generated
  \item inverse transformations are easy to read off by switching patterns for input and output
  \end{itemize}

  \subsection{Approaches}
  
  \begin{itemize}
  \item evidence based for API design, via real world use cases
  \item explicit separation of a few functions over 1 function, for better error messages
  \item consideration of adoption friction and ease of use
  \end{itemize}

  \vfill\null
  \columnbreak
    
  \subsection{Known Issues}
  \begin{itemize}
  \item does not enforce axes alignment between operations
  \item no means of integrated analysis/tracking of shapes
  \end{itemize}
  
  \vfill\null
  \pagebreak
  
\end{multicols*}
\end {document}


